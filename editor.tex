\chapter*{出版部報告}

この彙報を継続的に発行するために、\LaTeX を使ってコピー本を印刷するための方法を簡単に
述べておく。
まず、言うまでもなく\LaTeX のソースファイルを作成する。
その後、袋とじ状にして印刷するために、紙の片面に2頁分印刷するよう配置ししなければ
ならない。これは結構大変な作業である。

しかし、一旦PostScriptに変換し、PostScript関連ユーティリティを使うと簡単に
ページを配置することができる。

Debian GNU/LinuxやUbuntsuを使っているのであれば、psutilsパッケージをインストール
すればよい。
\begin{listing}
$ sudo apt-get install psutils 
\end{listing}
%$
%
インストールできたら、dvips コマンドでdviファイルをPostScriptに変換する。
\begin{listing}
$ dvips foo.dvi
\end{listing} 
%$
次にpsbookコマンドで袋とじ状にページを配置し、psnupコマンドで2頁を1頁にまとめて
ps2pdfコマンドでPDFに変換する。
% A4で出力するとき
\begin{listing}
psbook foo.ps |psnup -2 | ps2pdf - foo.pdf
\end{listing}
これだとA4の紙に2頁印刷されるので、A5のコピー本ができあがる。
A3の紙に印刷して、A4のコピー本を作るためのPDFを生成するには、psresizeコマンドで
2倍に拡大する。これだけでは用紙サイズの設定がA4のままなので、ps2psコマンドで
用紙サイズをA3に変更する。
% A3で出力する時
\begin{listing}
psbook foo.ps |psnup -2|psresize -pa3 |ps2ps -sPAPERSIZE=a3 - - | ps2pdf - foo.pdf
\end{listing}
これでA3サイズに2頁印刷するPDFができる。しかし、時々生成されたPDFファイルの向きが90度ずれることがあり、印刷に失敗する可能性がある。
A4に2頁印刷する場合はこのようなことがないので、A4用のPDFを作成し、プリンタの設定でA3に拡大するほうが確実だろう。

印刷は、PDFなのでほとんどのプリンタで印刷できるだろう。
PostScriptを直接印刷できるプリンタが使えるなら、ps2pdfでPDFにする必要はない。
本研究所はプリンタを所有していないので、本彙報はセブンイレブンのネットプリントを使って
印刷した。
