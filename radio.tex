\chapter{無線部報告}
ここでは、無線関係、特に電磁波、交流理論の非常に基礎的な部分で私が学習していて気になった部分について述べる。教科書に載っている程度のレベルの話であり、読者がこの程度のことを理解しているのであれば、確実に退屈するであろう。また、冗長ではあるがなるべく計算の過程を省略せずに書く。これは私が後に振り返った時にどうしてそうなったかを思い出すためである。もしも読者が学習途上にあるなら、自分の手を動かして、私が辿った過程をなぞってみることをお勧めする。

\section{単相交流の実効値の導出}
無線工学の教科書には、交流電圧の実効値は振幅の$\sqrt{2}$分の1であると書かれているが、導出まで
述べられていないことがある。まずはこれを導出してみよう。

交流電圧が正弦波交流と仮定すると、電圧$V$は、振幅を$V_0$とすると以下のように書ける。
\begin{equation}
V=V_0 \sin(\omega t) \label{eq:defv}
\end{equation}
ここで$\omega$とは角振動数で単位時間あたりの位相の変化率である。角振動数は、周波数を$f$とすると
\[
\omega = 2 \pi f
\]
であり、振動の周期を$T$とすると
\[
T = \frac{1}{f} = \frac{2 \pi}{\omega}
\]
である。交流電圧の実効値$V_e$は、1周期あたりの平均の電圧である。$\sin$関数を1周期の間で積分すると$0$になってしまうため、$V$を二乗して積分し、その結果の平方根を実効値とする。すなわち
\begin{equation}
V_e^2 =\frac{1}{T} \int_{0}^{T}V_0^2\sin^2(\omega t)dt \label{eq:Veff_sq}
\end{equation}
$\sin$の二乗の積分はすぐにはできないので、倍角公式を使って$\cos 2\omega t$の積分に変換する。
倍角公式は、
\[
\sin^2\theta = \frac{1-\cos 2\theta}{2}
\]
であるから、式(\ref{eq:Veff_sq})に上式を代入して積分を実行すると
\begin{eqnarray*}
V_e^2 &=& \frac{1}{T}\int_{0}^{T} V_0 \frac{1-\cos 2\omega t}{2} dt \\
&=& \frac{V_0^2}{2T}\int_0^{T} 1 - \cos 2\omega t dt \\
&=& \frac{V_0^2}{2T}[t + \frac{1}{2 \omega}\sin 2\omega t ]_0^{T} \\
&=& \frac{V_0^2}{2}
\end{eqnarray*}
両辺の平方根をとり、符号が正のものを実効値とすると、
\[
V_e = \frac{V_0}{\sqrt{2}}
\]
となる。

\section{正弦波交流の複素表示}
式(\ref{eq:defv})で、時刻$t$における正弦波交流の電圧の瞬時値$V_0$を定義したが、
Eularの公式
\[
e^{j\theta} = \cos\theta + j\sin\theta
\]
を用いて
\begin{equation}
V = V_e e^{j\omega t}
\end{equation}
と書く。ここで$j=\sqrt{-1}$である。電圧の計算値が必要な場合は、この虚部をとれば良い。
このように書く理由は、正弦波が作る電場、磁場の時間微分の計算が以下のように容易になるためである。

$\lambda$を定数として、関数$f(t)=e^{\lambda t}$の時間微分は
\[
\frac{df}{dt}=\lambda e^{\lambda t}
\]
であるから、たとえば正弦波交流が作る電場の大きさを$E=E_0e^{j\omega t}$とすると、この時間微分は
\[
\frac{dE}{dt} = E_0 j\omega e^{j \omega t} = j\omega E
\]
となり、微分を定数$j\omega$との積に変換することができる。

また、複素平面に書いた時、実部は位相に相当する。

\section{Eularの公式の証明}

\section{マクスウェル方程式から電磁波を導出}
