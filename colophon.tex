\chapter*{編集後記}
雑音製作所技術部彙報 vol.1をお届けします。ってもう誰も読んでいないかな。

この同人誌は、私が所属しているsoubrissive-devというサークルで、アキバケットなるイベントに出ることになったから
なんか作れと言われたことから始まりました。
仕事上、大電力RFを扱うこともあり、無線の資格だけでも極めようということで、第一級陸上無線技術士
を取ろうと目論んでいました。そんなときにこの話がきたので、これ幸いとネタにしたしだいです。

この内容は、私が勉強に使ったノートのようなものです。後日読み返しても思考の過程がたどれるように
計算の経過をあえて残してあります。基本的に教科書に書いてあるレベルのことなので、RFを知っている人にはつまらない、
知らない人にはなんのこっちゃわからん内容になっています。
とはいえ資格取得用の無線工学の教科書ではスルーされている内容を扱っているので、少しは意味があったかなと
(勝手に)思っています。

作っている方としては、締切りが設定されている状況下ゆえに、勉強しなきゃいけない状況を作り出す絶好の材料になりました。
また、(極小数とはいえ)人様の目に触れるものなので、恥ずかしくない内容にしようと、結構勉強を強いられます。
就職して仕事をするようになって、学生のころに基礎の部分を勉強しなかった現実を突きつけらました。
私の怠惰な性格もあり、日常業務にかまけて勉強しなおすきっかけが、なかなかつかめないでいました。
今回、同人誌作成が勉強するきっかけになったことで、勉強した内容を同人誌としてまとめ、頒布する
同人誌勉強法として流行らせたいと密かに思っております(笑)

どうでもいいことですが、本当は結城先生の「数学ガール」のような語り口にしたかったのは秘密です。
でも、体験したことのない甘酸っぱい恋心を創作する能力がなかったことは、もっとヒミツです。
次回がもしあれば、これに挑戦...するかどうかは未定ですが、伝送線路の話とかやりたいなと思っています。
あとコンピュータのハード側の話や、ネットワークの話もしたいと思ってます。

最後に、この同人誌はすべてオープンソースソフトウェアを使って作成されました。Debian/GNU Linux、\TeX Live2012
psutils、git、Gummiなどなど。また日本語のフォントはIPAexフォントを埋め込んで使っています。
このような有益なソフトウェアを開発、維持、管理していただいているすべての皆様に感謝します。
また、このページまでたどり着いてくれた読者の方(おそらくあなただけです)に感謝します。
ありがとうございました。

\begin{flushright}
2013年7月 Chomy(@jm6xxu) 拝\\
\end{flushright}

\subsection*{参考文献}
\begin{itemize}
  \item 中島将光
    「マイクロ波工学」 森北出版 ISBN4-627-71030-5
  \item 常川光一
    「線状アンテナから電波が出るしくみ」 CQ出版社 RFワールド No.11 pp.30-39, ISBN978-4-7898-4890-9
\end{itemize}