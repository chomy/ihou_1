\chapter*{編集後記}
雑音製作所偽述部彙報 vol.1をお届けします。ってもう誰も読んでいないかな。

この同人誌は、私が所属しているsoubrissive-devというサークルで、「アキバケットなるイベントに出ることになったから
なんか作れ」と言われたことから始まりました。
仕事上、大電力RFを扱うこともあり、第一級陸上無線技術士を取ろうと目論んでいました。
そんなときにこの話がきたので、これ幸いとネタにしたしだいです。

締切りが設定されている状況下ゆえに、勉強しなきゃいけない状況を作り出す絶好の材料になりました。
また、(極小数とはいえ)人様の目に触れるものなので、恥ずかしくない内容にしようとすると、結構な勉強が必要になります。
就職して学生のころに基礎の部分を勉強していなかったという現実を、厳しく突きつけられたにもかかわらず、
日常業務の忙しさを言い訳に、なかなか勉強を始めるきっかけがつかめないでいました。
今回うまくいったことで気を良くし、勉強した内容を同人誌としてまとめ、頒布する、「同人誌勉強法」として流行らせたいと密かに思っております(笑)

この内容は、私が勉強に使ったノートのようなものです。後日読み返しても思考の過程がたどれるように
計算の経過をあえて残してあります。基本的に教科書に書いてあるレベルのことなので、RFを知っている人にはつまらない、
知らない人にはなんのこっちゃわからん内容になっています。
とはいえ資格取得用の無線工学の教科書ではスルーされている内容を扱っているので、少しは意味があったかなと
(勝手に)思っています。

どうでもいいことですが、本当は結城先生の「数学ガール」のような語り口にしたかったのは秘密です。
しかし体験したことのない、甘酸っぱい恋心を創作する能力がなかったことはもっとヒミツです。
次回がもしあれば、これに挑戦...するかどうかはわかりませんが、伝送線路の話とかやりたいなと思っています。
あとコンピュータのハード側の話や、ネットワークの話もしたいと思ってます。

最後に、この同人誌はすべてオープンソースソフトウェアを使って作成されました。Debian/GNU Linux、\TeX Live2012
psutils、git、Gummiなどなど。また日本語のフォントはIPAexフォントを埋め込んだPDFを印刷しています。
このような有益なソフトウェアを開発、維持、管理していただいているすべての皆様に感謝します。
また、このページまでたどり着いてくれた読者の方(おそらくあなただけです)に感謝します。
ありがとうございました。

\begin{flushright}
2013年7月 Chomy(@jm6xxu) 拝
\end{flushright}

\subsection*{参考文献}
\begin{itemize}
  \item 中島将光
    「マイクロ波工学」 森北出版 ISBN4-627-71030-5
  \item 常川光一
    「線状アンテナから電波が出るしくみ」 CQ出版社 RFワールド No.11 pp.30-39, ISBN978-4-7898-4890-9
\end{itemize}